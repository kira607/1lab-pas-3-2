%%%%%%%%%%%%%%%%%%%%% General stuff %%%%%%%%%%%%%%%%%%%%%%%%%%%

\documentclass[12pt,a4paper]{article}  % шаблон для статьи, шрифт 12 пт

\usepackage[warn]{mathtext} % для отображения кириллицы в формулах (НУЖНО загружать ДО fontenc и babel)

\usepackage[utf8]{inputenc}  % использование кодировки Юникод UTF-8
\usepackage[T1]{fontenc}
\usepackage[russian]{babel}  % пакет поддержки русского языка

\usepackage{indentfirst}  % отступ первого абзаца
\setlength{\parindent}{0.75cm}

\usepackage[compact]{titlesec}  % для titlespacing
% \titlespacing{\заголовок}{слева}{перед}{после}[справа]
\titlespacing*{\section}{0.75cm}{1em}{0.1em}  % отступ заголовка
\titlespacing*{\subsection}{0.75cm}{1em}{0.1em}

%%%%%%%%%%%%%%%%%%%%% Pictures %%%%%%%%%%%%%%%%%%%%%%%%%%%

\usepackage{graphicx}  % кртинки
\usepackage{float} % плавающие картинки
\usepackage{wrapfig}  % Обтекание фигур (таблиц, картинок и прочего)
\usepackage[labelsep=endash]{caption}  % тире вместо двоеточия в картинках
\usepackage{subcaption} % place more than 1 figure at a picture

%%%%%%%%%%%%%%%%%%%%% Math symbols %%%%%%%%%%%%%%%%%%%%%%%%%%%

% \DeclareSymbolFont{T2Aletters}{T2A}{cmr}{m}{it} % переопределение шрифта для кириллицы в формулах на курсив
\usepackage{amsmath}

%%%%%%%%%%%%%%%%%%%%% Code %%%%%%%%%%%%%%%%%%%%%%%%%%%

\usepackage{xcolor}
\usepackage{listings}  % листинги кода из файлов

\definecolor{codegreen}{rgb}{0,0.6,0}
\definecolor{codegray}{rgb}{0.5,0.5,0.5}
\definecolor{codepurple}{rgb}{0.58,0,0.82}
\definecolor{backcolour}{rgb}{0.95,0.95,0.92}

\lstdefinestyle{cpp}
{
    backgroundcolor=\color{backcolour},% цвет фона подсветки
    commentstyle=\color{codegreen},
    keywordstyle=\color{magenta},
    numberstyle=\small\color{codegray},% размер шрифта для номеров строк
    stringstyle=\color{codepurple},
    basicstyle=\ttfamily\footnotesize, % размер и начертание шрифта для подсветки кода
    breakatwhitespace=false,           % переносить строки только если есть пробел  
    breaklines=true,                   % автоматически переносить строки (да\нет)  
    captionpos=t,                      % позиция заголовка вверху [t] или внизу [b]
    keepspaces=true,
    numbers=left,                   % где поставить нумерацию строк (слева\справа)
    numbersep=5pt,                  % как далеко отстоят номера строк от подсвечиваемого кода
    showspaces=false,               % показывать или нет пробелы специальными отступами
    showstringspaces=false,         % показывать или нет пробелы в строках
    showtabs=false,                 % показывать или нет табуляцию в строках
    tabsize=2,                      % размер табуляции по умолчанию равен 2 пробелам
    language=c++,                   % выбор языка для подсветки (здесь это С++)
    stepnumber=1,                   % размер шага между двумя номерами строк
    frame=false,                    % рисовать рамку вокруг кода
    escapeinside={\%*}{*)},         % если нужно добавить комментарии в коде
    extendedchars=\true
}

\lstdefinestyle{pythonstyle}
{
    backgroundcolor=\color{backcolour},% цвет фона подсветки
    commentstyle=\color{codegreen},
    keywordstyle=\color{magenta},
    numberstyle=\small\color{codegray},% размер шрифта для номеров строк
    stringstyle=\color{codepurple},
    basicstyle=\ttfamily\footnotesize, % размер и начертание шрифта для подсветки кода
    breakatwhitespace=false,           % переносить строки только если есть пробел  
    breaklines=true,                   % автоматически переносить строки (да\нет)  
    captionpos=t,                      % позиция заголовка вверху [t] или внизу [b]
    keepspaces=true,
    numbers=left,                   % где поставить нумерацию строк (слева\справа)
    numbersep=5pt,                  % как далеко отстоят номера строк от подсвечиваемого кода
    showspaces=false,               % показывать или нет пробелы специальными отступами
    showstringspaces=false,         % показывать или нет пробелы в строках
    showtabs=false,                 % показывать или нет табуляцию в строках
    tabsize=2,                      % размер табуляции по умолчанию равен 2 пробелам
    language=python,                % выбор языка для подсветки
    stepnumber=1,                   % размер шага между двумя номерами строк
    frame=false,                    % рисовать рамку вокруг кода
    escapeinside={\%*}{*)},         % если нужно добавить комментарии в коде
    extendedchars=\true
}

\lstdefinelanguage
[x64]{Assembler}     % add a "x64" dialect of Assembler
[x86masm]{Assembler} % based on the "x86masm" dialect
% with these extra keywords:
{morekeywords={PUSH, POP, LDI, SBI, CBI}} % etc.

\lstdefinestyle{myasm}
{
    backgroundcolor=\color{backcolour},% цвет фона подсветки
    commentstyle=\color{codegreen},
    keywordstyle=\color{magenta},
    numberstyle=\small\color{codegray},% размер шрифта для номеров строк
    stringstyle=\color{codepurple},
    basicstyle=\ttfamily\footnotesize, % размер и начертание шрифта для подсветки кода
    breakatwhitespace=false,           % переносить строки только если есть пробел  
    breaklines=true,                   % автоматически переносить строки (да\нет)  
    captionpos=t,                      % позиция заголовка вверху [t] или внизу [b]
    keepspaces=true,                 
    numbers=left,                   % где поставить нумерацию строк (слева\справа)            
    numbersep=5pt,                  % как далеко отстоят номера строк от подсвечиваемого кода           
    showspaces=false,               % показывать или нет пробелы специальными отступами       
    showstringspaces=false,         % показывать или нет пробелы в строках
    showtabs=false,                 % показывать или нет табуляцию в строках        
    tabsize=2,                      % размер табуляции по умолчанию равен 2 пробелам
    language=[x64]Assembler,             % выбор языка для подсветки             
    stepnumber=1,                   % размер шага между двумя номерами строк
    frame=false,                    % рисовать рамку вокруг кода
    escapeinside={\%*}{*)},         % если нужно добавить комментарии в коде
    extendedchars=\true
}

\lstset{style=pythonstyle}

%%%%%%%%%%%%%%%%%%%%% Other %%%%%%%%%%%%%%%%%%%%%%%%%%%

\usepackage{comment}

%\usepackage[showframe=true]{geometry} % shows the borders of the pages content
\usepackage{changepage} % Для смещения таблиц влево